\documentclass[a4paper,12pt]{article} 
\usepackage[T2A]{fontenc}			
\usepackage[utf8]{inputenc}			
\usepackage[english,russian]{babel}	
\usepackage{amsmath,amsfonts,amssymb,amsthm,mathtools} 
\usepackage[colorlinks, linkcolor = blue]{hyperref}
\usepackage{upgreek}\usepackage[left=2cm,right=2cm,top=2cm,bottom=3cm,bindingoffset=0cm]{geometry}
\usepackage{multirow}
\usepackage{graphicx,wrapfig,lipsum}
\usepackage{xcolor}
\usepackage{pgfplots}


\graphicspath{ {images/} }
\usepackage{multicol}
\setlength{\columnsep}{2cm}


\begin{document}

\begin{titlepage}
	\centering
	\vspace{5cm}
	{\scshape\LARGE Московский физико-технический институт \par}
	\vspace{4cm}
	{\scshape\Large Лабораторная работа 3.1.3 \par}
	\vspace{1cm}
	{\huge\bfseries Измерение магнитного поля Земли. \par}
	\vspace{1cm}
	\vfill
\begin{flushright}
	{\large выполнила студентка ФРКТ \par}
	\vspace{0.3cm}
	{\LARGE Миндиярова Рената}
\end{flushright}
	

	\vfill

% Bottom of the page
	Долгопрудный, 2021 г.
\end{titlepage}

\textbf{Цель работы:} определить характеристики шарообразных неодимовых магнитов и, используя
законы взаимодействия магнитных моментов с полем, измерить горизонтальную и вертикальную
составляющие индукции магнитного поля Земли и магнитное наклонение.\\
\textbf{В работе испольются:}
12 одинаковых неодимовых магнитных шариков, тонкая нить для
изготовления крутильного маятника, медная проволока диаметром (0,5 – 0,6) мм, электронные весы,
секундомер, измеритель магнитной индукции АТЕ-8702, штангенциркуль, брусок из немагнитного
материала (25*30*60 мм3
), деревянная линейка, штатив из немагнитного материала;
дополнительные неодимовые магнитные шарики (~ 20 шт.) и неодимовые магниты в форме параллелепипедов (2 шт.), набор гирь и разновесов.
\section*{Теория}
\subsection*{Точечный магнитный диполь}
Простейший магнитный диполь может быть образован витком с током или постоянным магнитом. По определению, магнитный момент $\overrightarrow{P_m}$ тонкого витка площадью $S$ с током $I$ равен
$$
\overrightarrow{P_m}=\dfrac{I}{c}\vec{S}=\dfrac{I}{c}S\vec{n},
$$
где $\vec{S}=S\vec{n}$ -- вектор площади круга контура. Если размеры контура с током или магнитной стрелки малы по сравнению расстоянием до диполя, то соответствующий магнитный диполь называют элементарным или точечным.\\
Магнитное поле точечного диполя определяется по формуле, аналогичной формуле для поля
элементарного электрического диполя:
$$
\vec{B}=\dfrac{3(\overrightarrow{P_m},\vec{r})\vec{r}}{r^5} - \dfrac{\overrightarrow{P_m}}{r^3}
$$ 
В магнитном поле с индукцией $B$
на точечный магнитный диполь 
действует механический
момент сил:
$$
\vec{M} = \overrightarrow{P_m}\times \vec{B}.
$$
Под действием вращающего момента $\vec{M}$ виток с током или постоянный магнит поворачивается
так, чтобы его магнитный момент выстроился вдоль вектора индукции магнитного поля. Это —
положение устойчивого равновесия: при отклонении от этого положения возникает механический
момент внешних сил, возвращающий диполь к положению равновесия. В положении, когда $\overrightarrow{P_m}$ и $\vec{B}$
параллельны, но направлены противоположно друг другу, также имеет место равновесие ($M$ = 0),
но такое равновесие неустойчиво: малейшее отклонение от этого положения приведёт к появлению
момента сил, стремящихся отклонить диполь ещё дальше от начального положения.\\
Магнитный диполь в магнитном поле обладает энергией:
$$
W = -(\overrightarrow{P_m},\vec{B})
$$
В неоднородном поле на точечный магнитный диполь, кроме момента сил, действует ещё и сила:
$$
\vec{F}=(\overrightarrow{P_m},\vec{\triangledown})\vec{B}
$$
Используя формулы для момента силы, силы и энергии, не сложно выяснить, как ведёт себя
свободный магнитный диполь в неоднородном магнитном поле: он выстраивается вдоль силовых
линий магнитного поля и, кроме того, под действием результирующей силы, возникающей из-за
неоднородности поля, втягивается в область более сильного магнитного поля, т.е. в область, где он
обладает меньшей энергией.\\
Зная магнитные моменты $P_1 = P_2 = P_m$ двух небольших постоянных магнитов, можно рассчитать силу
их взаимодействия:
$$
F = P_m \dfrac{\partial B}{\partial r}=-6\dfrac{P_m^2}{r^4}.
$$
\subsection*{Неодимовые магнитные шары}
В настоящей работе используются неодимовые магниты шарообразной формы.
Для нас важно то, что:\\
1) шары намагничены однородно;\\
2) вещество, из которого изготовлены магниты, является магнитожёстким материалом.\\
Полный магнитный момент $\overrightarrow{P_m}$
постоянного магнита определяется намагниченностью $\overrightarrow{p_m}$
вещества, из которого он изготовлен. По определению, намагниченность – это магнитный момент единицы объёма. Для однородно намагниченного шара намагниченность равна:
$$
\overrightarrow{p_m}=\dfrac{\overrightarrow{P_m}}{V}.
$$
Намагниченность — важная характеристика вещества постоянных магнитов, определяющая, в
частности, величину остаточной магнитной индукции $B_r = 4\pi p_m$. Индукция магнитного поля $\overrightarrow{B_p}$
на полюсах однородно намагниченного шара связана с величиной намагниченности и остаточной магнитной индукцией формулами
$$
\overrightarrow{B_p}=\dfrac{8\pi}{3}\overrightarrow{p_m}=\dfrac{2}{3}\overrightarrow{B_r}.
$$
\section*{Описание работы}
\subsection*{Определение величины магнитного момента магнитных шариков}
\paragraph*{Метод А}
Величину магнитного момента одинаковых шариков
можно рассчитать, зная их массу $m$ и определив максимальное расстояние $r_{max}$, на котором они ещё удерживают друг
друга в поле тяжести. При максимальном расстоянии сила тяжести шариков равна силе их магнитного притяжения:
\begin{center}
$\dfrac{6P_m^2}{r_{max}^4}=mg\Rightarrow$ \fbox{$P_m = \sqrt{\dfrac{mgr_{max}^4}{6}}$}
\end{center}
\paragraph*{Метод Б}
Если сила сцепления двух одинаковых шаров диаметром $d$ c магнитными моментами $P_m$ равна:
$$
F_0 = \dfrac{6P_m^2}{d^4}
$$
то минимальный вес цепочки, при которой она оторвётся от верхнего шарика равен: $F \approx 1.08 F_0$. Тогда\\
\begin{center}
\fbox{$P_m = \sqrt{\dfrac{Fd^4}{6.48}}$}
\end{center}
\subsection*{Определение величины магнитного поля Земли}
\paragraph*{Горизонтальная составляющая}
\begin{wrapfigure}{R}{3.5cm}
\includegraphics[scale=0.5]{1.png}
\vspace{-60pt}
\end{wrapfigure}  
Магнитная <<стрелка>> образована из $n$ сцепленных друг с другом противоположными полюсами шариков и с помощью $\Lambda$-образного подвеса подвешена в горизонтальном положении. При отклонении «стрелки» на угол $\theta$ от равновесного положения в горизонтальной плоскости возникают крутильные колебания вокруг вертикальной оси, проходящей через середину стрелки. При малых амплитудах уравнение колебаний
стрелки имеет вид:
$$
I_n \dfrac{d^2 \theta}{dt^2} + P_0 B_h \theta = 0,
$$ 
где $P_0$ -- магнитный момент стрелки, $B_h$ -- горизонтальная составляющая магнитного поля Земли, $I_n \approx \dfrac{1}{12}n^3 m d^3$, тогда период колебаний $T = kn$, где $k = \pi \sqrt{\dfrac{md^2}{3P_m B_h}}$. Измеряя зависимость $T=T(n)$, находится $B_h$:
\begin{center}
\fbox{$B_h = \dfrac{\pi^2 m d^2}{3k^2P_m}$}
\end{center}
\begin{wrapfigure}{R}{6cm}
\includegraphics[scale=0.5]{2.png}
\end{wrapfigure} 
\paragraph*{Вертикальная составляющая} 
Магнитная «стрелка», составленная из чётного числа
шариков и подвешенная на тонкой нити за середину, расположится не горизонтально, а под некоторым, отличным от нуля, углом к горизонту. Это связано с тем, что вектор $\vec{B}$ индукции магнитного поля Земли в общем случае не горизонтален, а образует с горизонтом
угол $\beta$, зависящим от географической широты $\varphi$
места, где проводится опыт. Величина угла $\beta$
называется магнитным наклонением.\\
С помощью небольшого дополнительного грузика «стрелку» можно «выровнять». Момент $M$ силы тяжести уравновешивающего груза пропорционален числу $n$ шариков, образующих магнитную «стрелку» $M(n) = An, A=P_m B_v$, то есть
\begin{center}
\fbox{$B_v = \dfrac{A}{P_m}$}
\end{center}
\section*{Ход работы}
\begin{enumerate}
\item Определим параметры шариков: масса 12 штук $m_{12} = 5.763\pm 0.001~\text{г}$, тогда масса одного $m = (4.802 \pm 0.001)*10^{-1}~\text{г}$, диаметр $d = 4.9 \pm 0.01~\text{мм}$.\\
Для измерения $r_max$ поместим шарики по разные стороны книги так, чтобы они всё ещё притягивались, и будем подкладывать листы бумаги между ними до тех пора, пока сила тяжести не пересилит силу притяжения. Измеренный $r_{max} = 1.97 \pm 0.01~\text{см}$. Тогда 
Магнитное поле однородно намагниченого шара внутри него : $B_o = \dfrac{2*P_m}{R^3}$
\begin{center}
$P_m = 34.4 \pm 0.4 ~\text{СГС}$
\end{center}



Составим цепочку из $25$ шариков, с помощью неодимовых магнитов в форме параллелепипедов, подсоединим цепочку к гире и разновесам так, чтобы общая масса системы составила приблизительно $500 \;г$. Далее подберём минимальный вес системы цепочки с гирей, при котором она отрывается от верхнего шарика.
Взвесим оторвавшуюся цепочку с гирей. $$m_{min} = 223.65 \pm 0.05 \; г$$
Рассчитаем силу сцепления двух шаров и по ней определим магнитный момент шарика $\mathfrak{m}$. 
\[ F_0 = \frac{6P_m^2}{d^4} \] \[F = m_{min}g  = F_0(1+\frac1{2^4}+\frac1{3^4} +...)\approx 1.08 F_0 \]
\[ P_m = \sqrt{\frac{d^4m_{min}g}{6 \cdot 1.08}} \;\]

$P_m = 10.8 \pm 0.22 ~\text{СГС}$.

Полученные значения магнитных моментов отличаются. Это может быть связано с большой погрешностью методики эксперимента, а так же неточным взаимным расположением магнитных моментов из-за силы трения. 
За итоговый момент возьмем  \begin{center}
	$P_m = 34.4 \pm 0.4 ~\text{СГС}$
	\end{center}
	
\item Соберём установку для измерения горизонтальной составляющей магнитного поля Земли. Перед непосредственным измерение проверим, можно ли пренебречь упругостью нити при измерении периода колебаний. Для этого сделаем из шариков кольцо, чтобы магнитный момент был нулевым, и посмотрим на его период колебаний $T = 40 \pm 0.1~\text{с}$. Такой большой период колебаний указывает на то, что упругостью можно пренебречь.\\

Соберём крутильный маятник из 12 магнитных шариков и подвесим его на немагнитном штативе. Используя $\Lambda$- образный подвес, установим "магнитную стрелку" в горизонтальное положение.
Возбудим крутильные колебания маятника вокруг вертикальной оси и определим их период. Исследуем зависимость периода $T$ крутильных колебаний "стрелки" от количества магнитных шариков n, составляющих "стрелку". 

\begin{table}[h]
\centering
\begin{tabular}{|l|l|l|l|l|l|l|l|l|l|l|}
\hline
$n$ & 12    & 11    & 10    & 9     & 8     & 7     & 6     & 5     & 4    & 3     \\ \hline
$T$, с & 3.34 & 3.08 &  2.83   & 2,58 & 2,32 & 2,08  & 1,82 & 1,54 & 1,22  & 0,96 \\ \hline
\end{tabular}
\end{table}
\begin{figure}[ht!]
	\centering
	\includegraphics[width=0.8\linewidth]{graf1.pdf}
	\label{C}
\end{figure}

Полученный коэффциент наклона $k= 0.263pm 0.001~\text{с}$. Тогда 
$$B_{\|} = \frac{\pi^2 md^2n^2}{3 T_n^2 P_m}\approx 0,16 \pm 0.008 ~\text{СГС} $$


\item Для магнитных <<стрелок>> произведём уравновешивание, измерив $M(n)$:
\begin{table}[h]
	\centering
\begin{tabular}{|l|l|l|}
\hline
n  &  m   & M \\ \hline
12 & 0.212 & 242,02      \\ \hline
10 & 0.140  & 211,29      \\ \hline
8  & 0.125  & 170,13     \\ \hline
6  & 0.125  & 118,61       \\ \hline
4  & 0.115 & 92,68       \\ \hline
\end{tabular}
\end{table}

\begin{center}
	\begin{figure}[ht!]
		\centering
		\includegraphics[width=0.8\linewidth]{graf2.pdf}
		\label{C}
	\end{figure}
	 
\end{center}
Полученный коэффциент наклона $a = 19,6 \pm 4,3$. Тогда
$$M_n = nP_mB_{\perp} \Rightarrow B_{\perp} = \frac{a}{P_m} \approx 0.57 \pm 0.03 ~\text{СГС} $$
 
\item Итоговая индукция магинтного поля Земли \fbox{$B = \sqrt{B_{\perp}^2 +B_{\|} ^2} = 0.59 \pm 0.13~\text{Гс}$}.
Магнитное наклонение \fbox{$\beta= \text{arctg}~\dfrac{B_{\perp}}{B_{\|}}=74^\circ\pm 2^\circ$}. 
\end{enumerate}
\section*{Выводы и рассчет погрешностей}
\subsection*{Погрешности}
$$ \varepsilon_{P_m1} = \sqrt{\Big(\frac{\Delta m}m\Big)^2 + 4\Big(\frac{\Delta r_{max}}{r_{max}}\Big)^2} \approx 6 \%$$
$$ \varepsilon_{P_m2} = \sqrt{\Big(\frac{\Delta d}d\Big)^2 + 4\Big(\frac{\Delta r_{min}}{r_{min}}\Big)^2} \approx 2 \%$$
\[\varepsilon_{B_{\|}} = \sqrt{\Big(\frac{\Delta m}m\Big)^2 + \Big(\frac{\Delta d}d\Big)^2 + \Big(\frac{\Delta P_m}{P_m}\Big)^2 + \Big(\frac{\Delta \frac{T_n}{n}}{\frac{T_n}{n}}\Big)^2} \approx 5 \% \]
\[\varepsilon_{B_{\perp}} = \sqrt{(\Big(\frac{\Delta a}a\Big)^2 + \Big(\frac{\Delta P_m}{P_m}\Big)^2} \approx 6 \%\]
\subsection*{Вывод}
Поскольку  установка находится в железобетонном
здании, магнитное поле в нём сильно отличается от поля Земли. Так же на
показания влияет наличие электронных приспособлений связи. Магнитное
поле Земли в нашем районе около 0.05 - 0.1 ед. СГС. Полученное значение не сильно отличается действительного.  
Используя результаты измерений $B_{\perp]}$ и $B_{\|}$, магнитное наклонение $\beta$ и полная величина индукции магнитного поля Земли в кабинете выполнения лабораторной работы равны. 
$$\beta = \arctan{\frac{B_{\perp}}{B_{\|}}} \approx 74^{\circ}  $$



\end{document}