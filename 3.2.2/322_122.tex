\documentclass[a4paper,12pt]{extreport}
\usepackage[T2A]{fontenc}			
\usepackage[utf8]{inputenc}			
\usepackage[english,russian]{babel}	
\usepackage{amsmath,amsfonts,amssymb,amsthm,mathrsfs,mathtools} 
\usepackage{cancel}
\usepackage{multirow}
\usepackage[colorlinks, linkcolor = blue]{hyperref}
\usepackage{upgreek}\usepackage[left=2cm,right=2cm,top=2cm,bottom=3cm,bindingoffset=0cm]{geometry}
\usepackage{graphicx,wrapfig,subfig}
\usepackage{xcolor}
\usepackage{graphicx}
\usepackage{csvsimple}
\usepackage{multirow}



\begin{document}

\begin{titlepage}
	\centering
	\vspace{5cm}
	{\scshape\LARGE Московский физико-технический институт \par}
	\vspace{4cm}
	{\scshape\Large Лабораторная работа 3.2.2 \par}
	\vspace{1cm}
	{\huge\bfseries Резонанс напряжений в последовательном контуре  \par}
	\vspace{1cm}
	\vfill
\begin{flushright}
	{\large выполнила студентка группы Б01-007}\par
	\vspace{0.3cm}
	{\LARGE Миндиярова Рената}
\end{flushright}
\vfill

% Bottom of the page
	Долгопрудный, 2021 г.
\end{titlepage}

	\section*{Аннотация}
	
	\hspace{\parindent}\textbf{Цель работы:} исследование резонанса напряжений в последовательном колебательном контуре с изменяемой ёмкостью, включающее получение амплитудно-частотных и фазово-частотных характеристик,а также определение основных параметров контура.
	
	\textbf{В работе используются:} генератор сигналов,источник напряжения,нагруженный на последовательный колебательный контур с переменной ёмкостью,двулучевой осциллограф, цифровые вольтметры.
	
	\section*{Описание работы}
\begin{center}
\includegraphics[width = 0.5\textwidth]{2.png}
\end{center}
Схема экспериментального стенда для изучения резонанса напряжений в последовательном колебательном контуре показана на рисунке. Синусоидальный сигнал от генератора GFG8255A поступает через согласующую RC-цепочку на вход источника напряжения, собранного на операционном усилителе ОУ. Питание операционного усилителя осуществляется
встроенным блоком-выпрямителем от сети переменного тока 220 Вольт (цепь питания на
схеме не показана). Источник напряжения, обладающий по определению нулевым внутренним сопротивлением, фактически обеспечивает с высокой точностью постоянство амплитуды сигнала на меняющейся по величине нагрузке – последовательном колебательном контуре, изображенном на рисунке в виде эквивалентной схемы.
\section*{Ход работы}
\begin{enumerate}
\item Подготавливаем установку к работе и включаем приборы.
\item Выставляем на входе контура напряжение $E = 150~\text{мВ}$, в течении всей работы поддерживая его постоянным.
\item Добиваемся получения двух отцентрованных синусоид на осциллографе. Убеждаемся, что одна из синусоид при изменении частоты $f$ генератора меняет амплитуду относительно начала координа, в то время как амплитуда другой не меняется с погрешностью не более 1\%.
\item Для контуров с семью различными ёмкостями, меняя их с помощью переключателя на
блоке, измеряем резонансные частоты $f_{0n}$ и напряжения $U_C(f_{0n})$. Регистрируйем также
напряжения $E(f_{0n})$, игнорируя отклонения в пределах относительной погрешности 1%.
\item Для контуров ёмкостями $C_1 = 47.6~\text{нФ}$ и $C_1 = 102.8~\text{нФ}$ снимаем амплитудно-частотные характеристики $U_C(f)$ (16-17 точек
в сумме по обе стороны от резонанса) при том же напряжении $E$.
\begin{table}[h]
\centering
\begin{tabular}{ccccc||c|c|c|c|c|}
\hline
\multicolumn{5}{|c||}{$C = 47.6$ нФ}                                                                                                         & \multicolumn{5}{c|}{$C = 102,8$ нФ}        \\ \hline
\multicolumn{1}{|c|}{$n$} & \multicolumn{1}{c|}{$f$, кГц} & \multicolumn{1}{c|}{$\sigma_f$, кГц} & \multicolumn{1}{c|}{$A$, В} & $\sigma_A$, В         & $n$ & f, кГц & $\sigma_f$, кГц & A, В & $\sigma_A$, В \\ \hline
\multicolumn{1}{|c|}{1}   & \multicolumn{1}{c|}{21,94}     & \multicolumn{1}{c|}{0,1}             & \multicolumn{1}{c|}{1,11}   & 0,01                  & 1   & 15,1    & 0,1             & 1,23 & 0,01          \\ \hline
\multicolumn{1}{|c|}{2}   & \multicolumn{1}{c|}{22,13}     & \multicolumn{1}{c|}{0,1}             & \multicolumn{1}{c|}{1,25}   & 0,01                  & 2   & 15,17   & 0,1             & 1,3 & 0,01          \\ \hline
\multicolumn{1}{|c|}{3}   & \multicolumn{1}{c|}{22,18}       & \multicolumn{1}{c|}{0,1}             & \multicolumn{1}{c|}{1,29}   & 0,01                  & 3   & 15,17   & 0,1             & 1,32 & 0,01          \\ \hline
\multicolumn{1}{|c|}{4}   & \multicolumn{1}{c|}{22,29}     & \multicolumn{1}{c|}{0,1}             & \multicolumn{1}{c|}{1,4}   & 0,01                  & 4   & 15,25   & 0,1             & 1,4 & 0,01          \\ \hline
\multicolumn{1}{|c|}{5}   & \multicolumn{1}{c|}{22,68}       & \multicolumn{1}{c|}{0,1}             & \multicolumn{1}{c|}{1,93}   & 0,01                  & 5   & 15,31     & 0,1             & 1,49 & 0,01          \\ \hline
\multicolumn{1}{|c|}{6}   & \multicolumn{1}{c|}{22,81}     & \multicolumn{1}{c|}{0,1}             & \multicolumn{1}{c|}{2,18}   & 0,01                  & 6   & 15,38   & 0,1             & 1,6 & 0,01          \\ \hline
\multicolumn{1}{|c|}{7}   & \multicolumn{1}{c|}{23,02}     & \multicolumn{1}{c|}{0,1}             & \multicolumn{1}{c|}{2,6}   & 0,01                  & 7   & 15,49   & 0,1             & 1,76 & 0,01          \\ \hline
\multicolumn{1}{|c|}{8}   & \multicolumn{1}{c|}{23,15}     & \multicolumn{1}{c|}{0,1}             & \multicolumn{1}{c|}{2,8}   & 0,01                  & 8   & 15,57   & 0,1             & 1,84 & 0,01          \\ \hline
\multicolumn{1}{|c|}{9}   & \multicolumn{1}{c|}{23,39}     & \multicolumn{1}{c|}{0,1}             & \multicolumn{1}{c|}{2,84}   & 0,01                  & 9   & 15,66     & 0,1             & 1,97  & 0,01          \\ \hline
\multicolumn{1}{|c|}{10}  & \multicolumn{1}{c|}{23,43}     & \multicolumn{1}{c|}{0,1}             & \multicolumn{1}{c|}{2,79}   & 0,01                  & 10  & 15,7   & 0,1             & 2 & 0,01          \\ \hline
\multicolumn{1}{|c|}{11}  & \multicolumn{1}{c|}{23,55}     & \multicolumn{1}{c|}{0,1}             & \multicolumn{1}{c|}{2,63}   & 0,01                  & 11  & 15,73   & 0,1             & 2,04 & 0,01          \\ \hline
\multicolumn{1}{|c|}{12}  & \multicolumn{1}{c|}{23,64}     & \multicolumn{1}{c|}{0,1}             & \multicolumn{1}{c|}{2,51}   & 0,01                  & 12  & 15,75   & 0,1             & 2,04  & 0,01          \\ \hline
\multicolumn{1}{|c|}{13}  & \multicolumn{1}{c|}{23,78}       & \multicolumn{1}{c|}{0,1}           & \multicolumn{1}{c|}{2,27}   & 0,01                  & 13  & 15,88   & 0,1             & 2,05 & 0,01          \\ \hline
\multicolumn{1}{|c|}{14}  & \multicolumn{1}{c|}{23,8}     & \multicolumn{1}{c|}{0,1}             & \multicolumn{1}{c|}{2,2}   & 0,01                  & 14  & 15,95   & 0,1             & 2,02 & 0,01          \\ \hline
\multicolumn{1}{|c|}{15}  & \multicolumn{1}{c|}{23,9}     & \multicolumn{1}{c|}{0,1}             & \multicolumn{1}{c|}{1,97}   & 0,01                  & 15  & 16   & 0,1             & 1,98 & 0,01          \\ \hline
\multicolumn{1}{|c|}{16}  & \multicolumn{1}{c|}{24,12}     & \multicolumn{1}{c|}{0,1}             & \multicolumn{1}{c|}{1,75}   & 0,01                  & 16  & 16,08   & 0,1             & 1,88 & 0,01          \\ \hline
\multicolumn{1}{|c|}{17}  & \multicolumn{1}{c|}{24,15}     & \multicolumn{1}{c|}{0,1}             & \multicolumn{1}{c|}{1,7}   & 0,01                  & 17  & 16,22   & 0,1             & 1,72 & 0,01          \\ \hline
\multicolumn{1}{|c|}{18}  & \multicolumn{1}{c|}{24,47}     & \multicolumn{1}{c|}{0,1}             & \multicolumn{1}{c|}{1,36}   & 0,01                  & 18  & 16,44   & 0,1             & 1,43 & 0,01          \\ \hline
\multicolumn{1}{|c|}{19}  & \multicolumn{1}{c|}{24,59}     & \multicolumn{1}{c|}{0,1}             & \multicolumn{1}{c|}{1,26}   & 0,01                  & 19  & 16,58   & 0,1             & 1,27 & 0,01          \\ \hline
\end{tabular}
\end{table}
\newpage
\item Для тех же двух контуров снимите фазово-частотные характеристики $\varphi_C(f)$ (16-17 точек в сумме по обе стороны от резонанса) при том же
напряжении $E$.
\begin{table}[h]
\centering
\begin{tabular}{ccc||c|c|c|}
\hline
\multicolumn{3}{|c||}{$C = 47,6 $ нФ} & \multicolumn{3}{c|}{$C = 102,8$ нФ} \\ \hline
\multicolumn{1}{|c|}{$n$} & \multicolumn{1}{c|}{$f$, кГц} & $-\varphi/ \pi$ & $n$ & $f$, кГц & $-\varphi/ \pi$ \\ \hline
\multicolumn{1}{|c|}{1} & \multicolumn{1}{c|}{21,94} & 0,03 & 1 & 15,1 & 0,07 \\ \hline
\multicolumn{1}{|c|}{2} & \multicolumn{1}{c|}{22,13} & 0,04 & 2 & 15,17 & 0,08 \\ \hline
\multicolumn{1}{|c|}{3} & \multicolumn{1}{c|}{22,18} & 0,05 & 3 & 15,17 & 0,08 \\ \hline
\multicolumn{1}{|c|}{4} & \multicolumn{1}{c|}{22,29} & 0,05 & 4 & 15,25 & 0,11 \\ \hline
\multicolumn{1}{|c|}{5} & \multicolumn{1}{c|}{22,68} & 0,1 & 5 & 15,31 & 0,15 \\ \hline
\multicolumn{1}{|c|}{6} & \multicolumn{1}{c|}{22,81} & 0,14 & 6 & 15,38 & 0,17 \\ \hline
\multicolumn{1}{|c|}{7} & \multicolumn{1}{c|}{23,02} & 0,21 & 7 & 15,49 & 0,22 \\ \hline
\multicolumn{1}{|c|}{8} & \multicolumn{1}{c|}{23,15} & 0,29 & 8 & 15,57 & 0,27 \\ \hline
\multicolumn{1}{|c|}{9} & \multicolumn{1}{c|}{23,39} & 0,4 & 9 & 15,66 & 0,34 \\ \hline
\multicolumn{1}{|c|}{10} & \multicolumn{1}{c|}{23,43} & 0,49 & 10 & 15,7 & 0,36 \\ \hline
\multicolumn{1}{|c|}{11} & \multicolumn{1}{c|}{23,55} & 0,6 & 11 & 15,73 & 0,39 \\ \hline
\multicolumn{1}{|c|}{12} & \multicolumn{1}{c|}{23,64} & 0,71 & 12 & 15,75 & 0,4 \\ \hline
\multicolumn{1}{|c|}{13} & \multicolumn{1}{c|}{23,78} & 0,8 & 13 & 15,88 & 0,48 \\ \hline
\multicolumn{1}{|c|}{14} & \multicolumn{1}{c|}{23,8} & 0,83 & 14 & 15,95 & 0,57 \\ \hline
\multicolumn{1}{|c|}{15} & \multicolumn{1}{c|}{23,9} & 0,87 & 15 & 16 & 0,69 \\ \hline
\multicolumn{1}{|c|}{16} & \multicolumn{1}{c|}{24,12} & 0,93 & 16 & 16,08 & 0,79 \\ \hline
\multicolumn{1}{|c|}{17} & \multicolumn{1}{c|}{24,15} & 0,94 & 17 & 16,22 & 0,86 \\ \hline
\multicolumn{1}{|c|}{18} & \multicolumn{1}{c|}{24,47} & 0,98 & 18 & 16,44 & 0,93 \\ \hline
\multicolumn{1}{|c|}{19} & \multicolumn{1}{c|}{24,59} & 1 & 19 & 16,58 & 1 \\ \hline

 \end{tabular}
\end{table}
\end{enumerate}
\newpage
\section*{Обработка данных}
\begin{enumerate}
\item Результаты измерений представим в таблице.\\
\begin{table}[h]
\centering
\begin{tabular}{|c|c|c|c|c|c|c|c|c|c|c|c|}
\hline
$n$ & $C_n$, нФ & \begin{tabular}[c]{@{}c@{}}$f_{0n}$, \\ кГц\end{tabular} & $U_C$, В & $E$, В & \begin{tabular}[c]{@{}c@{}}$L$, \\ мкГн\end{tabular} & $Q$ & \begin{tabular}[c]{@{}c@{}}$\rho$, \\ Ом\end{tabular} & \begin{tabular}[c]{@{}c@{}}$R_{\sum}$, \\ Ом\end{tabular} & \begin{tabular}[c]{@{}c@{}}$R_{S_{\max}}$,\\ Ом\end{tabular} & $R_L$, Ом & $I$, мА \\ \hline
1 & 24,8 & 32,2 & 3,75 & 0,49 & 986,643 & 25,17 & 199,46 & 7,92 & 0,2 & 4,22 & 0,0080 \\ \hline
2 & 33,2 & 27,8 & 3,33 & 0,49 & 988,219 & 22,35 & 172,53 & 7,72 & 0,17 & 4,05 & 0,0093 \\ \hline
3 & 47,6 & 23,25 & 2,89 & 0,49 & 985,435 & 19,4 & 143,88 & 7,42 & 0,14 & 3,77 & 0,0110 \\ \hline
4 & 57,5 & 21,16 & 2,66 & 0,48 & 984,876 & 17,97 & 130,88 & 7,28 & 0,13 & 3,65 & 0,0121 \\ \hline
5 & 68 & 19,47 & 2,48 & 0,48 & 983,648 & 16,76 & 120,23 & 7,17 & 0,12 & 3,56 & 0,0132 \\ \hline
6 & 81,6 & - &  & - & - & - & - & - & - & - & - \\ \hline
7 & 102,8 & 15,79 & 2,06 & 0,48 & 989,289 & 13,91 & 98,1 & 7,05 & 0,10 & 3,45 & 0,0161 \\ \hline
\multicolumn{5}{|c|}{Среднее значение} & 986,35 & \multicolumn{4}{c|}{--} & 3,78 & -- \\ \hline
\multicolumn{5}{|c|}{\begin{tabular}[c]{@{}c@{}}Среднеквадратичная погрешность\\ среднего значения\end{tabular}} & 0,87 & \multicolumn{4}{c|}{--} & 0,12 & -- \\ \hline
\end{tabular}
\end{table}

$$f_0 =\frac{1}{2\pi }\cdot \frac{1}{\sqrt{LC}}$$
$$L = \frac{1}{4\pi^{2}}\cdot \frac{1}{\sqrt{f_{0}^{2}}C}$$
$$Q = \frac{U_c(\omega _0)}{\varepsilon _0(\omega _0)}$$
$$\rho = \sqrt{\frac{L}{C}}$$
$$R_{\sum} = \frac{\rho }{Q}$$
$$R_{S_{\max}}=10^{-3}\rho $$
$$R_{L} = R_{\sum} - R - R_{S_{\max}}$$
$$I_{max} = \frac{\varepsilon _0}{R_{\sum}}$$

\item По данным из пункта 5 построим на одном графике амплитудо-частотные характеристики в координатах $f, U_C(f)$.
\begin{center}
	\begin{figure}[ht!]
		\centering
		\includegraphics[width=0.8\linewidth]{achkh1.pdf}
		\label{C}
	\end{figure}
	 
\end{center}
\item По тем же данным построим на одном графике амплитудо-частотные характеристики в безразмерных координатах $x = f/f_{0n}, y = U_C(x)/U_C(1)$. 
\begin{center}
	\begin{figure}[ht!]
		\centering
		\includegraphics[width=0.8\linewidth]{achkhsliniey.pdf}
		\label{C}
	\end{figure}
	 
\end{center}
По ширине резонансных кривых по уровню 0.707 определим добротность Q соответствующих контуров.


C1:
$\Delta f = 0,0498 $кГц
$Q = 20$


C2:
$\Delta f = 0,0725 $кГц
$Q = 13.8$


Посчитаем погрешность резонансной частоты.
C1: $\sigma_{f_0} = \frac{f_{i+1} - f_{i}}{2} = \frac{23.43-23.15}{2} =  0.14 кГц$


C2: $\sigma_{f_0} = \frac{f_{i+1} - f_{i}}{2} = \frac{15.95-15.75}{2} =  0.1 кГц$
$$\delta_{f_0} = \frac{\sigma_{f_0}}{f_0} \approx 0.006$$
$$\sigma_{Q} = \delta_{f_0} \cdot Q$$
$$ Q_{C1} = 20\pm 0.12$$
$$ Q_{C1} = 13.8\pm 0.1$$


\item По данным пункта 6 построим на одном графике фазово-частотные характеристики в координатах $x = f/f_{0n}, y = \varphi /\pi$ для выбранных контуров. По этим характеристикам определим добротности контуров одним из двух способов: по расстоянию между
точками по оси $x$, в которых $y$ меняется от $-0.25$ до $-0.75$, равному $1/Q$, или по формуле $Q = 0.5~d\varphi_C(x)/dx$ при $x=0$: $Q_1 = 19 \pm 0,15$ и $Q_2 = 12,3 \pm 0,4$.
\begin{center}
	\begin{figure}[ht!]
		\centering
		\includegraphics[width=0.8\linewidth]{fchkh.pdf}
		\label{C}
	\end{figure}
	 

\end{center}
\item По данным таблицы построим зависимость $R_L(f_{0n})$, на график нанесём прямую $\langle R_L \rangle$.
\begin{center}
	\begin{figure}[ht!]
		\centering
		\includegraphics[width=0.8\linewidth]{Rl.pdf}
		\label{C}
	\end{figure}
	 
\end{center}
\item По данным построим векторную диаграмму тока
и напряжений для контура с наименьшей добротностью в резонансном состоянии. Ось
абсцисс направим по вектору $\vec{E}$.
\begin{center}
	\begin{figure}[ht!]
		\centering
		\includegraphics[width=0.8\linewidth]{vector.jpg}
		\label{C}
	\end{figure}
	 
\end{center}
\end{enumerate}
\end{document}