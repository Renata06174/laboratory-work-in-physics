\documentclass[a4paper,12pt]{article} 
\usepackage[T2A]{fontenc}			
\usepackage[utf8]{inputenc}			
\usepackage[english,russian]{babel}	
\usepackage{amsmath,amsfonts,amssymb,amsthm,mathrsfs,mathtools} 
\usepackage{cancel}
\usepackage{multirow}
\usepackage[colorlinks, linkcolor = blue]{hyperref}
\usepackage{upgreek}\usepackage[left=2cm,right=2cm,top=2cm,bottom=3cm,bindingoffset=0cm]{geometry}
\usepackage{graphicx,wrapfig,subfig}
\usepackage{xcolor}
\usepackage{graphicx}
\usepackage{csvsimple}
\usepackage{multirow}

\author{Миндиярова Р.В\\
Группа Б01-007}
\title{3.5.1. Изучение плазмы газового разряда в неоне.}
\date{}
\begin{document}
\maketitle
\textbf{Цель работы}: изучение вольт-амперной характеристики тлеющего разряда, изучение свойств плазмы методом зондовых характеристик.


\textbf{В работе используются}: стеклянная газоразрядная трубка, наполненная изотопом неона, высоковольтный источник питания (ВИП), источник питания постоянного тока, делитель напряжения, резистор, потенциометр, амперметры, вольтметры, переключатели.
\section*{Теория}
\subsection*{Плазма}
В ионизированном газе поле ионов <<экранируется>> электронами. Для поля $\mathbf{E}$ и плотности $\rho$ электрического заряда
$$
\text{div}~\mathbf{E} = 4 \pi \rho,
$$
а с учётом сферической симметрии и $\mathbf{E} = -\text{grad}~\varphi$:
\begin{equation}
\dfrac{d^2 \varphi}{dr^2}+\dfrac{2}{r}\dfrac{d\varphi}{dr}=-4\pi \rho.
\end{equation}
Плотности заряда электронов и ионов (которые мы считаем бесконечно тяжёлыми и поэтому неподвижными)
\begin{equation}
\begin{array}{c}
\rho_e = -ne \cdot \exp\left(\dfrac{e\varphi}{kT_e}\right),\\
\rho_i = ne.
\end{array}
\end{equation}
Тогда из $(1)$ в предположении $\dfrac{e\varphi}{kT_e} \ll 1$ получим
\begin{equation}
\varphi = \dfrac{Ze}{r}e^{-r/r_D},
\end{equation}
где $r_D = \sqrt{\dfrac{kT_e}{4\pi n e^2}}$ -- \textit{радиус Дебая}. Среднее число ионов в сфере такого радиуса 
\begin{wrapfigure}{r}{4cm}
\includegraphics[scale=0.5]{2.png}
\end{wrapfigure}  
\begin{equation}
N_D = n\dfrac{4}{3}\pi r_D^2.
\end{equation}
Теперь выделим параллелепипед с плотностью $n$ электронов, сместим их на $x$. Возникнут поверхностные заряды $\sigma = nex$, поле от которых будет придавать электронам ускорение:
$$
\dfrac{d^2x}{dt^2}=-\dfrac{eE}{m}=-\dfrac{4\pi n e^2}{m}x.
$$ 
Отсюда получаем \textit{плазменную (ленгмюровскую) частоту} колебаний электронов:
\begin{equation}
\omega_p = \sqrt{\dfrac{4\pi ne^2}{m}}.
\end{equation}
\subsection*{Одиночный зонд}
При внесении в плазму уединённого проводника -- \textit{зонда} -- с потенциалом, изначально равным потенциалу точки плазмы, в которую его помещают, на него поступают токи электроннов и ионов:
\begin{equation}
\begin{array}{c}
I_{e0} = \dfrac{n \langle v_e \rangle}{4}eS,\\
I_{i0} = \dfrac{n \langle v_i \rangle}{4}eS,
\end{array}
\end{equation}
где $\langle v_e \rangle$ и $\langle v_i \rangle$ -- средние скорости электронов и ионов, $S$ -- площадь зонда, $n$ -- плотность электронов и ионов. Скорости электронов много больше скорости ионов, поэтому $I_{i0} \ll I_{e0}$. Зонд будет заряжаться до некоторого равновестного напряжения $-U_f$ -- \textit{плавающего потенциала}.\\
\begin{wrapfigure}{r}{5.5cm}
\includegraphics[scale=0.5]{3.png}
\end{wrapfigure}  
В равновесии ионный ток мало меняется, а электронный имеет вид
$$
I_e = I_0 \exp\left( -\dfrac{eU_f}{kT_e} \right).
$$
Будем подавать потенциал $U_\text{з}$ на зонд и снимать значение зондового тока $I_\text{з}$. Максимальное значение тока $I_{e\text{н}}$ -- электронный ток насыщения, а минимальное $I_{i\text{н}}$ -- ионный ток насыщения. Значение из эмпирической формулы Бомона:
\begin{equation}
I_{i\text{н}} = 0.4 neS \sqrt{\dfrac{2kT_e}{m_i}}.
\end{equation}
\subsection*{Двойной зонд}
Двойной зонд -- система из двух одинаковых зондов, расположенных на небольшом расстоянии друг от друга, между которыми создаётся разность потенциалов, меньшая $U_f$. Рассчитаем ток между ними вблизи $I=0$. При небольших разностях потенциалов ионные токи на оба зонда близки к току насыщения и компенсируют друг друга, а значит величина результирующего тока полностью связана с разностью электронных токов. Пусть потенциалы на зондах
$$
U_1 = -U_f + \Delta U_1,
$$
$$
U_2 = -U_f + \Delta U_2.
$$
Между зондами $U = U_2 - U_1 = \Delta U_2 - \Delta U_1$.
Через первый электрод
\begin{equation}
I_1 = I_{i\text{н}} + I_{e1} = I_{i\text{н}} - \dfrac{1}{4}neS\langle v_e\rangle \exp\left(-\dfrac{eU_f}{kT_e}\right)\exp\left(\dfrac{e\Delta U_1}{kT_e}\right)=I_{i\text{н}}\left(1 - \exp\left( \dfrac{e\Delta U_1}{kT_e} \right)\right).
\end{equation}
Аналогично через второй получим
\begin{equation}
I_2 = I_{i\text{н}}\left(1 - \exp\left( \dfrac{e\Delta U_2}{kT_e} \right)\right)
\end{equation}
  
Из $(7)$ и $(8)$ с учётом последовательного соединение зондов ($I_1 = -I_2 = I)$:
$$
\Delta U_1= \dfrac{kT_e}{e}\text{ln}\left(1 - \dfrac{I}{I_{i\text{н}}}\right)
$$
$$
\Delta U_2= \dfrac{kT_e}{e}\text{ln}\left(1 + \dfrac{I}{I_{i\text{н}}}\right)
$$

Тогда итоговые формулы для разности потенциалов и тока

\begin{equation}
U = \dfrac{kT_e}{e}\text{ln}\dfrac{1 - I/I_{i\text{н}}}{1 + I/I_{i\text{н}}}, 
I = I_{i\text{н}} \text{th}\dfrac{eU}{2kT_e}.
\end{equation}
Реальная зависимость выглядит несколько иначе и описывается формулой 
\begin{wrapfigure}{l}{7cm}
\includegraphics[scale=0.8]{4.png}
\vspace{+30pt}
\end{wrapfigure}
\begin{equation}
I = I_{i\text{н}} \text{th}\dfrac{eU}{2kT_e} + AU.
\end{equation}
Из этой формулы можно найти формулу для $T_e$: для $U=0$ мы найдём $I_{i\text{н}}$, продифференцируем в точке $U=0$ и с учётом $\text{th}~\alpha \approx \alpha$ при малых $\alpha$ и $A\rightarrow 0$ получим:
\begin{equation}
kT_e = \dfrac{1}{2}\dfrac{eI_{i\text{н}}}{\dfrac{dI}{dU}|_{U=0}}.
\end{equation}
\section*{Описание установки}
\begin{center}
\includegraphics[scale=0.6]{1.png}
\end{center}
Стеклянная газоразрядная трубка имеет холодный (ненакаливаемый) полый катод, три анода и \textit{геттерный} узел -- стеклянный баллон, на внутреннюю повехность которого напылена газопоглощающая плёнка (\textit{геттер}). Трубка наполнена изотопом неона $^22$Ne при давлении 2 мм рт. ст. Катод и один из анодом (I и II) с помощью переключателя $\Pi_1$ подключается через балластный резистор $R_\text{б}$ ($\approx 450$ кОм) к регулируемому ВИП с выкодным напряжением до 5 кВ.\\
При подключении к ВИП анода-I между ним и катодом возникает газовый разряд. Ток разряда измеряется миллиамперметром $A_1$, а падение напряжения на разрядной трубке -- цифровым вольтметром $V_1$, подключённым к трубке черезе высокоомный (25 МОм) делитель напряжения с коэффициентом $(R_1+R_2)/R_2 = 10$.\\
При подключении к ВИП анода-II разряд возникает в пространстве между катодом и анодом-II, где находятся двойной зонд, используемый для диагностики плазмы положительного столба. Зонды изготовлены из молибденовой проволоки диаметром $d = 0.2$ мм и имеют длину $l = 5.2$ мм. Они подключены к источнику питания GPS через потенциометр $R$. Переключатель $\Pi_2$ позволяет изменять полярность напряжения на зондах. Величина напряжения на зондах изменяеься с помощью дискретного переключателя <<$V$>> выходного напряжения источника питания и потенциометра $R$, а измеряется цифровым вольтметром $V_2$. Для измерения зондового тока используется мультиметр $A_2$.





\section{Выполнение работы}

Данные установки $R_b = $ 450 $\: kOm, d = $ 0.2 $\: mm, l =$  5.2 $\: mm$

 \subsection{Вольт-амперная характеристика разряда}

Снимаем значения, составляем таблицу 1:
\begin{table}[!]
    \caption{Таблица c измерениями}
        \begin{center}
            \begin{tabular}{| c | c |}
            \hline
            \textbf{I, mkA} & \textbf{V, volt}\\
            \hline
            4,8	& 26,83\\
            \hline
            4,6 & 26,8\\
            \hline
            3,4	& 27\\
            \hline 
            3,2	& 27,1\\
            \hline
            2,96 &	27,3\\
            \hline
            2,6	& 27,9\\
            \hline
            2,16 &	29\\
            \hline
            1,8	& 30\\
            \hline
            1,4	& 31,8\\
            \hline
            1 & 32,8\\
            \hline
            0,95 &	33,2\\
            \hline
        \end{tabular}
    \end{center}
\label{A_table}
\end{table}


По полученным значениям строим график зависимости U(I) - вольт-амперную характеристику разряда. 

\begin{figure}[h!]
	\centering
	\includegraphics[width=0.8\linewidth]{1_ris_351.pdf}
	\caption{Вольт-амперная характеристика разряда}
	\label{C}
\end{figure}

По наклону касательной к графику определим максимальное дифференциальное сопротивление разряда $R_{max} \approx 3.46 \: kOm \:$:
\subsection{Работа с зондом}
Снимаем значения, составляем таблицу 2:
\begin{table}[]
    \caption{}
    \centering
    \label{tab:my-table}
    \begin{tabular}{|l|l|l|l|l|l|}
        \hline
    \textbf{V1, volt} & \textbf{I1, mkA} & \textbf{V2, volt} & \textbf{I2, mkA} & \textbf{V3, volt} & \textbf{I3, mkA}\\
    95,3  & 25,2  & 55,1   & 25,3   & 25,1   & 24,15  \\ \hline
    96,7  & 22,28 & 53,1   & 22,07  & 24,26  & 21,02  \\\hline
    95,5  & 19,29 & 51,4   & 19,3   & 23,49  & 18,12  \\\hline
    92,62 & 16,39 & 49,28  & 15,97  & 22,64  & 15,08  \\\hline
    85,27 & 13,03 & 46,44  & 12,99  & 21,67  & 12,66  \\\hline
    74,03 & 10,17 & 42,56  & 10,67  & 18,45  & 9,1    \\\hline
    62,09 & 8,17  & 35,62  & 8,2    & 14,92  & 7,05   \\\hline
    47,31 & 6,26  & 26     & 6,14   & 10,25  & 5,15   \\\hline
    26,71 & 4,11  & 15,77  & 4,12   & 4,15   & 3,05   \\\hline
    3,48  & 2     & 1,31   & 1,99   & 0,9    & 2,1    \\\hline
    0,33  & 1,78  & -13,1  & -0,5   & -13,68 & -2,09  \\\hline
    -21,2 & -1,06 & -30,2  & -3,12  & -19,83 & -4,16  \\\hline
    -42,8 & -3,06 & -42,87 & -5,5   & -24,05 & -5,98  \\\hline
    -64,2 & -5,28 & -53,6  & -8,52  & -27,66 & -8,2   \\\hline
    -85,5 & -8,3  & -57,6  & -10,29 & -29,93 & -10,29 \\ \hline
    -94     & -10,1  & -61,9  & -13,13 & -31,83 & -13,1  \\ \hline
-104,5  & -13,1  & -64,3  & -16    & -33,32 & -16,35 \\\hline
-110,7  & -16,1  & -66,84 & -19    & -34,6  & -19,32 \\\hline
-114,11 & -19,31 & -68,42 & -21,54 & -35,82 & -22,14 \\\hline
-115,08 & -21,3  & -70,27 & -24,14 & -36,69 & -24,13 \\\hline
    \end{tabular}
    \end{table}



\begin{figure}[h!]
	\centering
	\includegraphics[width=0.8\linewidth]{current_5.pdf}
	\caption{График зондовой характеристик $i_{razr} = 5 \: mA$}
	\label{C}
\end{figure}

\begin{figure}[h!]
	\centering
	\includegraphics[width=0.8\linewidth]{currenT_3.pdf}
	\caption{График зондовой характеристик $i_{razr} = 3 \: mA$}
	\label{C}
\end{figure}
\begin{figure}[h!]
	\centering
	\includegraphics[width=0.8\linewidth]{last_graf.png}
	\caption{График зондовой характеристик $i_{razr} = 3 \: mA$}
	\label{C}
\end{figure}
\newpage
По графикам 2, 3, 4 рассчитаем $I_{насыщ}$, а так же $\frac{dI}{dU}, \: U = 0$.
Для этого проводим соответствующие ассимптоты и касательные. Сведем полученные результаты в таблицу 3:
	
\begin{table}[!]
    \caption{Таблица для расчетов}
    \begin{center}
        \begin{tabular}{|c|c|c|}
            \hline
            $I_{razr}, mA$ & $I_{iH}, mkA$ & $\frac{dI}{dU}, \frac{mkA}{V}$ \\
            \hline
        1.5 & 19 & 6.96  \\
            \hline
        3.0 & 39 & 32 \\
            \hline
        5.0 & 73 & 9.66 \\
            \hline
        \end{tabular}
    \end{center}
  \end{table}
  
  Учтем погрешности.

  \textbf{Теперь рассчитаем температуру электронов} $T_e$ по формуле (12), а также $n_e$ - концентрацию электронов в плазме по формуле Бома (7).
  
  \begin{table}[!]
    \caption{Таблица для расчетов}
    \begin{center}
        \begin{tabular}{|c|c|c|c|c|}
            \hline
        $ I_{razr}, $, mA  & $ kT_e $ , $el \cdot Volt $  & $n_e \cdot 10^{15}, m^{-3}$ & $T_e, K \cdot 10^{4}$ & $\sigma{T_e}  K \cdot 10^{4}$ \\
            \hline
        1.5 & 1.36 & 26.3 & 1.6 & 0.18 \\
            \hline
        3.0 & 0.64 & 44 & 0.7 & 0.08 \\
            \hline
        5.0 & 0.44 & 77 & 0.5 & 0.06 \\
            \hline
        \end{tabular}
    \end{center}
  \label{B_table}
  \end{table}
  Построим график зависимости $ n_{e} = f(I_{razr})$

  \begin{figure}[h!]
      \centering
      \includegraphics[width=1.1\linewidth]{пенис игоря.png}
      \caption{График зависимости $ n_{e}$ от $I_{razr}$}
      \label{C}
  \end{figure}
  
  
  \textbf{Затем рассчитаем плазменную частоту колебаний электронов} $\omega_e$, а так же дебаевский радиус экранирования (с учетом того, что температура ионов мала по сравнению с электронной). 
  \begin{table}[!]
    \caption{Таблица для расчетов}
    \begin{center}
        \begin{tabular}{|c|c|c|c|c|}
            \hline
        $I_{razr}$ mA & $\omega_p, \cdot 10^{11}, \frac{rad}{sec}$ & $r_D \cdot 10^{-2}, cm$ & $N_D $ & $\alpha \cdot 10^{-7}$ \\
            \hline
        1.5 & 0.87 & 0.21 &  387 &  4.60  \\
            \hline
      3.0 & 1.56 & 0.16  & 171 &  7.81  \\
            \hline
        5.0 & 2.18 & 0.13  & 92 & 13.6  \\
            \hline
        \end{tabular}
    \end{center}
  \label{B_table}
  \end{table}
  \textbf{Теперь оценим среднее число ионов} в дебаевской среде $N_D$. 
  Примем $r_D \approx 10^{-3} m$ судя из рассчетов. Тогда $R_D \approx 10^{8}$
  а также степерь ионизации плазмы долю ионизированных атомов $\alpha$ при учете, что давление в трубке $P \approx  2 \: Torr$.
  Сведем все полученные результаты в итоговую таблицу 5.
  
  
  \section{Вывод}
  
  В данной работе мы изучили вольт-амперную характеристику тлеющего разряда.
  Затем занялись изучением свойств плазмы методом зондовых характеристик.
  \newline
  В этом пункте мы получили, что температура электронов у нас порядка $T_e \approx 10^{4} \: K$, тогда $kT_e \approx 1 \: eV$. 
  \newline
  Концентрация электронов в плазме получилось порядка $n_e \approx 10^{16}$. 
  \newline
  Плазменная частота колебаний получилось порядка $\omega_p \approx 10^{16} \: \frac{rad}{sec}$.
  \newline
  Дебаевский радиуc получили $r_D \approx 10^{-3} \: m$, среднее число ионов в дебаевской сфере много больше единицы (см. таблицу 5). 
  
  Полученные значения близки к табличным. 
 
 
 \end{document}    

